\documentclass[conference]{IEEEtran}
\IEEEoverridecommandlockouts
% The preceding line is only needed to identify funding in the first footnote. If that is unneeded, please comment it out.
\usepackage{cite}
\usepackage{amsmath,amssymb,amsfonts}
\usepackage{algorithmic}
\usepackage{graphicx}
\usepackage{textcomp}
\usepackage{xcolor}
\def\BibTeX{{\rm B\kern-.05em{\sc i\kern-.025em b}\kern-.08em
    T\kern-.1667em\lower.7ex\hbox{E}\kern-.125emX}}
\begin{document}

\title{Conference Paper Title\\
}

\author{\IEEEauthorblockN{Aarya Gupta, Aditya Raj Jain, Aditya Kumar Singh}

\IEEEauthorblockA{\textit{dept. name of organization (of Aff.)} \\
\textit{Indraprastha Institute of Information Technology}\\
Delhi, India \\}
}

\maketitle

\begin{abstract}
Every Autonomous vehicle's safety is bound by its protective capabilities and decision-making process.\\ Probabilistic prediction varies depending on real-life factors. It can be divided into Aleatory(stochastic) and epistemic(non-stochastic). \\
This project aims to analyse various methods to improve the protective capabilities of a simple kinematic bicycle's steering angle prediction model.\\
We primarily compare and combine the KDE (Kernel Density Function) and Bootstrapping to predict the next angle while using Bayesian updation (Maximum A posteriori) to improve the model with the obtained data of previous iterations.
\end{abstract}

\begin{IEEEkeywords}
Autonomous vehicles, Kinematic bicycle model, Aleatory uncertainty
Epistemic uncertainty, Kernel Density Estimation (KDE),
Bootstrapping, Maximum A posteriori (MAP)
\end{IEEEkeywords}

\section{Introduction}
\textcolor{red}{Add introduction about all the methods used to predict the distribution and table explaining timeline.\\ }

The system operates under the following assumptions:

\begin{enumerate}
    \item \textbf{Discrete Time System}: The system is modelled as a discrete-time system.
    \item \textbf{Uniform Time Increments}: Time increments are fixed, with $\Delta t = 1$.
    \item \textbf{Time Indexing}: Time instants are indexed as \\ $t = 0, 1, 2, 3, \ldots$.
    \item \textbf{Underlying System Equations}: The bicycle model in our world follows the following equations:

    \begin{subequations}
    \renewcommand{\theequation}{1\alph{equation}}
    \begin{equation}
    \theta(t) = \theta(t-\Delta t) + \Delta \theta(t)
    \end{equation}
    \begin{equation}
    x(t) = x(t-\Delta t) + v \cdot \cos(\theta(t)) \cdot \Delta t
    \end{equation}
    \begin{equation}
    y(t) = y(t-\Delta t) + v \cdot \sin(\theta(t)) \cdot \Delta t
    \end{equation}
    \end{subequations}
    Here, 
    \begin{itemize}
        \item $\theta$ is the R.V. describing the heading angle of the vehicle.
        \item $\Delta\theta$ is the R.V. describing the change in heading angle.
        \item $x$ and $y$ are the R.V.s depicting the position.
    \end{itemize}

    \item \textbf{Initial Conditions and Assumptions}:
    \begin{itemize}
        \item Let the true underlying (unknown) distribution of $\Delta\theta(t)$ be $\hat{\mathcal{F}}$, which is same for all time instants.
        \item Values of $x_0$, $y_0$, $\theta_0$, $v$ are definite/constants and are given.
        \item For simplicity, let's assume $x_0$, $y_0$, $\theta_0$ to be 0 while the vehicle is moving with a constant speed of 1.
        \begin{equation}
        \Rightarrow \theta_0 = 0, \quad \Delta\theta_0 = 0, \quad x_0 = 0, \quad y_0 = 0, \quad v = 1
        \end{equation}
        \item Assuming the initial set of values $\mathcal{F}_{1}$ are given, from which $\Delta \theta_1$ will chose a value. $\mathcal{F}_{1}$ represents the distribution of $\Delta \theta_1$.
    \end{itemize}
    
\end{enumerate}

% \begin{table}[h!]
%     \centering
%     \small  % Reduce font size to fit the table within the column width
%     \begin{tabular}{|c|c|c|c|c|}
%         \hline
%         & $t=0$ & $t=1$ & $t=2$ & $t=3$ \\
%         \hline
%         Known RVs & \parbox[c]{2.5cm}{$\theta_0=0$, $\Delta \theta_0 = 0$ \\ $x_0=0$, $y_0=0$ \\ $\mathcal{F}_{1}$ $(\Delta \theta_1$ distrib.)}  & \parbox[c]{2.4cm}{$\Delta \theta_1$ (observed) \\ $\theta_1$, $x_1$, $y_1$} & \parbox[c]{2.4cm}{$\Delta \theta_2$ (observed) \\ $\theta_2$, $x_2$, $y_2$} & \parbox[c]{2.4cm}{$\Delta \theta_3$ (observed) \\ $\theta_3$, $x_3$, $y_3$}\\
%         \hline
%         Predicted RVs & - & $\mathcal{F}_{2}$ $(\Delta \theta_2$ dist.) & $\mathcal{F}_{3}$ $(\Delta \theta_3$ dist.) & $\mathcal{F}_{4}$ $(\Delta \theta_4$ dist.) \\
%         \hline
%     \end{tabular}
%     \vspace{0.2cm}  % Adjust vertical space as needed
%     \caption{System Timeline}
%     \raggedright
%     \label{tab:sample_table}
% \end{table}

% Table with columns till t=2 so that the table can be inserted completely into the page.
\begin{table}[h!]
    \centering
    \renewcommand{\arraystretch}{1.75} % Adjust the value to increase or decrease the vertical spacing
    \small  % Reduce font size to fit table within column width
    \begin{tabular}{|c|c|c|c|}
        \hline
        & $t=0$ & $t=1$ & $t=2$ \\
        \hline
        Known RVs & \parbox[c]{2.25cm}{\centering $\theta_0=0$, $\Delta \theta_0 = 0$ \\ $x_0=0$, $y_0=0$ \\ $\mathcal{F}_{1}$ $(\Delta \theta_1$ distrib.)}  & \parbox[c]{2.35cm}{\centering $\Delta \theta_1$ (observed) \\ $\theta_1$, $x_1$, $y_1$} & \parbox[c]{2.35cm}{\centering $\Delta \theta_2$ (observed) \\ $\theta_2$, $x_2$, $y_2$} \\
        \hline
        Predicted RVs & - & $\mathcal{F}_{2}$ $(\Delta \theta_2$ distr.) & $\mathcal{F}_{3}$ $(\Delta \theta_3$ distr.) \\
        \hline
        \parbox[c]{1.8cm}{\centering True Distrib.\\ (Unknown)} & $\hat{\mathcal{F}}$ & $\hat{\mathcal{F}}$ & $\hat{\mathcal{F}}$ \\
        \hline
    \end{tabular}
    \vspace{0.2cm}  % Adjust vertical space as needed
    \caption{System Timeline}
    \raggedright
    $\mathcal{F}_{t}$ is the predicted distribution at time $t-1$ from $\mathcal{F}_{t-1}$ $\forall$ $t \geq 2$. The true value of $\Delta \theta_t$ at $t$ will be observed from the true unknown distribution $\hat{\mathcal{F}}$.
    \label{tab:sample_table}
\end{table}

To estimate the Steering angle PDF at the next time step, we look at the following methods:
\subsection{Kernel Density Estimation (KDE)}
Imagine you have a bunch of data points, and you want
to understand their distribution. Kernel Density Estimation
(KDE) is like a sophisticated way of creating a smooth
curve that represents this distribution by placing a curve (Kernel) at all the points from the sample and adding them up to find the distribution (It is normalised such that probability adds up to 1). It’s particularly useful when you don’t know the underlying shape of your data’s distribution.

\subsection{Bootstrapping}
Initially, given a set of real-world samples, we create bootstrap samples at each time step by randomly selecting elements from the original set, allowing replacement. The predicted distribution at each time step is the bootstrapped distribution of the original set. The true value of the change at each time step is observed from this predicted distribution. We calculate the Mean Squared Error (MSE) at each time step to understand the underlying distribution of changes over time. This MSE is measured between the predicted and initial real-world data distributions. We aim to minimize this MSE as time progresses, indicating that our predictions are becoming more accurate and approaching the unknown underlying distribution.



\subsection{KDE combined with Bootstrapping}
In this, we combine the benefits of KDE and bootstrapping. From the original samples, we bootstrap it to produce 500 samples. For each sample, we construct a KDE distribution. All the samples are assigned a different bandwidth of the kernel. Finally, all the distributions are combined to form a final distribution, with a bandwidth normalised from all the samples. It increases the computational complexity but provides a better estimate of the underlying distribution.

\subsection{Bayesian Updation}
\textcolor{red}{To be discussed}

\section{Introduction to Kernel Density Estimation (KDE)}

\subsection{Basic Idea}

KDE works by placing a small "bump" (called a kernel) at each data point and then adding up all these bumps to create a smooth curve. This curve estimates the probability density function (PDF) of your data.

\subsection{Mathematical Formulation}

The kernel density estimator is defined as:

\begin{equation}
    \hat{f}_h(x) = \frac{1}{nh} \sum_{i=1}^n K\left(\frac{x - X_i}{h}\right)
\end{equation}

Here, $\hat{f}_h(x)$ is our estimated density at point $x$, $n$ is the number of data points, $h$ is the bandwidth (which controls how wide our bumps are), $K$ is the kernel function (the shape of our bumps), and $X_i$ are our individual data points.

\subsection{Kernel Functions}

The kernel function $K$ determines the shape of our bumps. It must integrate to 1 over its domain:

\begin{equation}
    \int_{-\infty}^{\infty} K(u) du = 1
\end{equation}

Common kernel functions include:

1. Gaussian (Normal) Kernel:
\begin{equation}
    K(u) = \frac{1}{\sqrt{2\pi}} e^{-\frac{1}{2}u^2}
\end{equation}

This is like using little bell curves as our bumps.

2. Epanechnikov Kernel:
\begin{equation}
    K(u) = \frac{3}{4}(1-u^2) \text{ if } |u|\leq 1, \text{ else } 0
\end{equation}

This uses parabolic bumps, which some consider optimal in a mathematical sense.

\section{The Math Behind KDE}

\subsection{Bias-Variance Tradeoff}

In statistics, we often deal with a tradeoff between bias (how far off our estimate is on average) and variance (how much our estimate varies with different samples). KDE is no exception.

The mean integrated squared error (MISE) helps us quantify this tradeoff:

\begin{equation}
    \text{MISE}(\hat{f}_h) = \int (\text{Bias}[\hat{f}_h(x)])^2 dx + \int \text{Var}[\hat{f}_h(x)] dx
\end{equation}

This equation tells us how good our estimate is overall, considering both bias and variance.

\subsection{Bandwidth Selection}

Choosing the right bandwidth $h$ is crucial. Too small, and our estimate will be too spiky; too large, and it will be too smooth, missing important features.

The optimal bandwidth that minimizes the asymptotic MISE is:

\begin{equation}
    h_{opt} = \left(\frac{R(K)}{n \mu_2(K)^2 R(f'')}\right)^{1/5}
\end{equation}

where $R(K) = \int K^2(u) du$, $\mu_2(K) = \int u^2 K(u) du$, and $R(f'') = \int (f''(x))^2 dx$.

In practice, we often use data-driven methods like cross-validation to choose $h$.

\subsection{Multivariate KDE}

When dealing with multiple variables (like position and velocity), we use multivariate KDE:

\begin{equation}
    \hat{f}_H(x) = \frac{1}{n} \sum_{i=1}^n K_H(x - X_i)
\end{equation}

Here, $H$ is a matrix that determines the width and orientation of our multidimensional bumps.

\section{Applying KDE to Your Data}

To use KDE on your incoming data:

1. Collect your data points (e.g., steering angles and accelerations).
2. Choose a kernel function (Gaussian is a good start).
3. Select a bandwidth (you can use cross-validation or a rule of thumb).
4. Apply the KDE formula to estimate the probability density.

This gives you a smooth estimate of the probability distribution of your data, which you can use for further analysis or prediction.

\section{Bootstrapping}
\subsection{Bootstrapping the Samples}

Given a set of real-world samples \( \{x_1, x_2, \ldots, x_n\} \), each bootstrap sample \(\mathbf{X}_i^*\) is defined as \( \{x_{i1}^*, x_{i2}^*, \ldots, x_{in}^*\} \) is obtained by randomly sampling from the original set with replacement. 
% We repeat this process 500 times to create 500 bootstrap samples \( \{X_1^*, X_2^*, \ldots, X_{500}^*\} \).


\subsection{Implying the above logic in our system (Refer to Table 1)}
% $\mathcal{F}_{t}$ is the bootstrapped distribution at time $t-1$ from $\mathcal{F}_{t-1}$ $\forall$ $t \geq 2$. The true value of $\Delta \theta_t$ at $t$ will be observed from the predicted $\mathcal{F}_{t}$ at t-1.

$\mathcal{F}_{t}$ is the bootstrapped distribution at time $t-1$ from $\mathcal{F}_{1}$ $\forall$ $t \geq 1$. The true value of $\Delta \theta_t$ at $t$ will be observed from the predicted $\mathcal{F}_{t}$ at t-1. \\
To reach the underlying distribution of $\Delta \theta_t$, we will find the \textbf{Mean Square Error} at time t defined as $\textbf{MSE}_\textbf{t}$, between predicted $\mathcal{F}_{t+1}$ \& $\mathcal{F}_{1}$; and we will try to minimise this MSE as time progresses.

The $\text{MSE}_{t}$ between two distributions (lists of values) $\mathcal{F}_{1}$ (given real-world data) \( \{x_1, x_2, \ldots, x_n\} \) and $\mathcal{F}_{t+1}$ \( \{\hat{x}_{t+1,1}, \hat{x}_{t+1,2}, \ldots, \hat{x}_{t+1,n}\} \) (data predicted at time t) is defined as:

\begin{subequations}
    \begin{equation}
    \text{MSE}_{t} = \frac{1}{n} \sum_{i=1}^{n} (x_i - \hat{x}_{t+1, i})^2
    \end{equation}
\end{subequations}

The aim is to minimise this MSE, and take it to below some predefined threshold value $\text{MSE}_{min}$ so that for some t, $\mathcal{F}_{t+1}$ is the distribution nearest to the unknown underlying distribution. 

% \textcolor{red}{More information to be added } 


\section{KDE combined with Bootstrapping}


\subsection{Creating KDE Distributions with Varying Bandwidths}
Once we have the bootstrapped samples, we create a KDE for each sample. The bandwidth of the KDE determines the smoothness of the estimated density. We explore two methods for selecting the bandwidth:

\subsection{Method 1: Assigning a Linear Range of Bandwidths}
In this method, we assign bandwidths to each KDE in a linearly increasing manner. Specifically, for the \( i \)-th bootstrapped sample, we set the bandwidth \( h_i \) as:
\[
h_i = \frac{c(i - 1)}{499}, \quad \text{for } i = 1, 2, \ldots, 500
\]
This means the first KDE has a bandwidth of 0, and the last has a bandwidth of c. Where c is decided on the basis of samples. \\
\textcolor{red}{Add information about the silverman and scott's rule and why this may be better than those}

\subsection{Method 2: Using the Standard Deviation as Bandwidth}
In the second method, we use the standard deviation of each bootstrapped sample as the bandwidth. For the \( i \)-th bootstrapped sample \( X_i^* \), the bandwidth \( h_i \) is:
\[
h_i = \text{std}(X_i^*)
\]
where \( \text{std} \) denotes the standard deviation of the sample \(X_i^*\)

\subsection{Combining KDE Distributions}
To create a final KDE distribution from the 500 individual KDEs, we average the estimated densities. If \( f_i(x) \) is the KDE for the \( i \)-th bootstrapped sample with bandwidth \( h_i \), the final combined KDE \( f(x) \) is:
\[
f(x) = \frac{1}{500} \sum_{i=1}^{500} f_i(x)
\]
Each \( f_i(x) \) is given by:
\[
f_i(x) = \frac{1}{n h_i} \sum_{j=1}^{n} K \left( \frac{x - x_j^*}{h_i} \right)
\]
where \( K \) is the kernel function, typically a Gaussian function:
\[
K(u) = \frac{1}{\sqrt{2\pi}} e^{-\frac{u^2}{2}}
\]

\section{Bayesian Updating with Maximum A Posteriori (MAP) Estimation}
When we have an existing distribution, and new data becomes available, we can update the existing distribution to reflect the new information using Bayesian updating. 
\subsection{Bayesian Updation}
\subsection{MAP Estimation}
Given a prior distribution \( p(\theta) \) and observed data \( D = \{x_1, x_2, \ldots, x_n\} \), the MAP estimate of the parameter \( \theta \) is the value that maximizes the posterior distribution \( p(\theta | D) \). Using Bayes' theorem, the posterior distribution is:
\[
p(\theta | D) \propto p(D | \theta) p(\theta)
\]
where \( p(D | \theta) \) is the likelihood of the data given the parameter \( \theta \).

The MAP estimate \( \hat{\theta}_{MAP} \) is given by:
\[
\hat{\theta}_{MAP} = \arg \max_\theta p(\theta | D)
\]

\subsection{Updating the KDE Distribution}
Suppose we have an initial KDE distribution \( f_{prior}(x) \) representing our prior knowledge of the distribution. After observing new data \( D = \{x_1, x_2, \ldots, x_n\} \), we can update this distribution using MAP estimation.

1. Construct the Likelihood: Assume the observed data \( D \) follows the KDE distribution \( f_{prior}(x) \). The likelihood \( p(D | f_{prior}) \) can be constructed from the KDE's evaluation on the new data points.

2. Update the Prior: Combine the prior KDE with the likelihood to get the posterior KDE. We achieve this by adjusting the KDE to incorporate the new data points. This can be done by re-estimating the KDE with the new data added to the original dataset or by weighting the prior and new KDEs.

3. Find the MAP Estimate: The MAP estimate in this context is the updated KDE that maximizes the posterior distribution.

\section*{Conclusion}

\textcolor{red}{Add conclusion}

\section{References}
\textcolor{red}{Add references}

\end{document}
